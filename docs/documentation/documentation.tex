\documentclass[10pt,a4paper]{article}
\usepackage[utf8]{inputenc}
\usepackage[francais]{babel}
\usepackage[T1]{fontenc}
\usepackage{amsmath}
\usepackage{amsfonts}
\usepackage{amssymb}
\author{Antoine Falaize}
\newcommand{\version}{0.1.5}
\title{\textsc{PyPHS Documentation}\\ Version \version}
\begin{document}
\renewcommand{\vec}[1]{\mathbf{#1}}
\newcommand{\mat}[1]{\mathbf{#1}}
\newcommand{\lab}[1]{\mathtt{#1}}
\newcommand{\op}[1]{\mathrm{#1}}

\newcommand{\Dpar}[2]{\frac{\partial #1}{\partial #2}}
\newcommand{\D}[2]{\frac{\op{d} #1}{\op{d} #2}}
\newcommand{\Dnth}[3]{\frac{{\op{d}}^{#3} #1}{\op{d} {#2}^{#3}}}
\newcommand{\T}{\intercal}

\renewcommand{\d}{\op{d}}

\DeclareSymbolFont{bbold}{U}{bbold}{m}{n}
\DeclareSymbolFontAlphabet{\mathbbold}{bbold}

\newcommand{\vx}{{\vec{x}}}
\newcommand{\vw}{{\vec{w}}}
\newcommand{\vz}{{\vec{z}}}
\newcommand{\vu}{{\vec{u}}}
\newcommand{\vy}{{\vec{y}}}
\newcommand{\vp}{{\vec{p}}}
\newcommand{\vg}{{\vec{g}}}
\newcommand{\vc}{{\vec{c}}}

\newcommand{\vvl}{{\vec{v}_{\lab l}}}
\newcommand{\vvnl}{{\vec{v}_{\lab{nl}}}}

\newcommand{\vfl}{{\vec{f}_{\lab l}}}
\newcommand{\vfnl}{{\vec{f}_{\lab{nl}}}}

\renewcommand{\H}{{\op{H}}}
\newcommand{\Z}[1]{\mat{Z}_{\lab{#1}}}
\newcommand{\J}[1]{\mat{J}_{\lab{#1}}}
\newcommand{\R}[1]{\mat{R}_{\lab{#1}}}
\newcommand{\M}[1]{\mat{M}_{\lab{#1}}}
\newcommand{\N}[1]{\mat{N}_{\lab{#1}}}
\newcommand{\dxH}[1]{\nabla\H_{\lab{#1}}}
\newcommand{\dxHd}[1]{\overline{\nabla}\H_{\lab{#1}}}
\newcommand{\Id}{\mat{I_d}}

\newcommand{\diag}{\operatorname{diag}}

\newcommand{\Q}{\mat{Q}}

\newcommand{\vxl}{\vx_{\lab l}}
\newcommand{\vxnl}{\vx_{\lab{nl}}}
\newcommand{\Hl}{\H_{\lab l}}
\newcommand{\Hnl}{\H_{\lab{nl}}}
\newcommand{\nx}{{n_{\lab{x}}}}
\newcommand{\nxl}{{n_{\lab{xl}}}}
\newcommand{\nxnl}{{n_{\lab{xnl}}}}

\newcommand{\vwl}{\vw_{\lab l}}
\newcommand{\vwnl}{\vw_{\lab{nl}}}
\newcommand{\vzl}{\vz_{\lab l}}
\newcommand{\vznl}{\vz_{\lab{nl}}}
\newcommand{\nw}{{n_{\lab{w}}}}
\newcommand{\nwl}{{n_{\lab{wl}}}}
\newcommand{\nwnl}{{n_{\lab{wnl}}}}

\newcommand{\e}{\mathfrak{e}}
\newcommand{\f}{\mathfrak{f}}
\newcommand{\ve}{\boldsymbol{\e}}
\newcommand{\vf}{\boldsymbol{\f}}

\newcommand{\ny}{n_{\lab{y}}}
\newcommand{\nc}{n_{\lab{y}}}
\renewcommand{\ng}{n_{\lab{g}}}
\newcommand{\np}{n_{\lab{y}}}

\newcommand{\ones}{ \mathbbold{1}}
\newcommand{\zeros}{ \mathbbold{0}}

\newcommand{\RR}{\mathbb{R}}
\newcommand{\CC}{\mathbb{C}}
\newcommand{\NN}{{\mathbb{N}}}
\newcommand{\ZZ}{{\mathbb{Z}}}


%%%%%%
%Fractional Calculus Commandes 
\newcommand{\bmu}{\boldsymbol{\mu}}
\newcommand{\Icx}{{\dot\iota}}
\newcommand{\HH}{{\mathbb{H}}}
\newcommand{\cC}{{\mathcal{C}}}
\newcommand{\EXP}[1]{{\mathrm{e}^{#1}}}
\newcommand{\myvector}[1]{{\mathbf{#1}}}%{{\underline{#1}}}
\newcommand{\Ev}[0]{\myvector{E}}
\newcommand{\Hv}[0]{\myvector{H}}
\newcommand{\Mv}[0]{\myvector{M}}
\newcommand{\Vv}[0]{\myvector{V}}
\newcommand{\Wv}[0]{\myvector{W}}
\newcommand{\herm}[1]{{\overline{#1}^\T}}
\newcommand{\npoles}{{n_\xi}}
\newcommand{\nfreqs}{{n_\omega}}
\newcommand{\fracint}{{\mathcal{I}}}
\newcommand{\fracdiff}{{\mathcal{D}}}

%
\maketitle
%
%
\section{Package structure}
%
Below is a list of each \textsc{PyPHS} module of practical use, along with a short description.
%
\begin{description}
%
\item[\texttt{symbs}] Container for all the \textsc{SymPy} symbolic variables. 
\begin{description}
%
\item[Attributes] are ordered \emph{list of symbols} associated with the system's vectors components:
%
\begin{description}
\item[\texttt{x}]: state vector symbols $\vx \in \RR^{\nx}$,
\item[\texttt{w}]: dissipative vector variable symbols $\vw\in\RR^{\nw}$,
\item[\texttt{u}]: input vector symbols $\vu\in\RR^{\ny}$,
\item[\texttt{y}]: output vector symbols $\vy \in \RR^{\ny}$,
\item[\texttt{cu}]: input vector symbols for connectors $\mathbf{c_u}\in\RR^{\nc}$,
\item[\texttt{cy}]: output vector symbols for connectors $\mathbf{c_y}\in\RR^{\nc}$,
\item[\texttt{p}]: Time-varying parameters symbols $\vp\in \RR^{\np}$.
\end{description}
%
\item[Methods]:
%
\begin{description}
%
\item[\texttt{dx()}]: Returns the symbols associated with the state differential $\d\vx$ formed by appending the prefix $d$ to each symbol in \texttt{x}.
%
\item[\texttt{args()}]: Return the list of symbols associated with the vector of all arguments of the symbolic expressions (\texttt{expr} module).
%
\end{description}
%
\end{description}
%
\item[\texttt{exprs}] Container for all the \textsc{SymPy} symbolic expressions associated with the system's functions. 
\begin{description}
%
\item[Attributes:] For scalar function (e.g. the storage function \texttt{H}), arguments are \textsc{SymPy} expressions; for vector functions (e.g. the disipative function $\vz$), arguments are ordered lists of \textsc{SymPy} expressions; for matrix functions (e.g. the Jacobian matrix of disipative function $\vz$), arguments are \texttt{sympy.Matrix} objects. Notice the expression arguments (\texttt{sympy.Symbols}) must belong to the following lists of the \texttt{pyphs.PorthamiltonianObject}: \texttt{symbs.x}, \texttt{symbs.dx()}, \texttt{symbs.w}, \texttt{symbs.u}, \texttt{symbs.p}, and the keys of the dictionary \texttt{symbs.subs}.
%
\begin{description}
\item[\texttt{H}]: storage function $\H\in\RR$,
\item[\texttt{z}]: dissipative function $\vz\in\RR^{\nw}$,
\item[\texttt{g}]: input/output gains vector function $\vg \in \RR^{\ng}$,
\end{description}
%
The following expression are computed from the \texttt{exprs.build()} metho (see below):
%
\begin{description}
\item[\texttt{dxH}]: the continuous gradient vector of storage scalar function $\dxH{}(\vx)$,
\item[\texttt{hessH}]: the continuous hessian matrix of storage scalar function (computed as $\nabla\nabla\H(\vx)$),
\item[\texttt{y}]: the expression of the continuous output vector function $\vy(\dxH{}, \vz, \vu)$,
\item[\texttt{dxHd}]: the discrete gradient vector of storage scalar function $\dxHd{}(\vx, \delta\vx)$,
\item[\texttt{yd}]: the expression of the discrete output vector function $\vy(\dxHd{}, \vz, \vu)$,
\item[\texttt{jacz}]: the continuous jacobian matrix of dissipative vector function $\nabla\vz(\vw)$.
\end{description}
%
\item[Methods]:
%
\begin{description}
%
\item[\texttt{build()}]: Build the following system functions as \textsc{SymPy} expressions and append them as attributes to the \texttt{exprs} module:
%
\begin{description}
\item[\texttt{dxH}]: the continuous gradient vector of storage scalar function $\dxH{}(\vx)\in\RR^{\nx}$,
\item[\texttt{dxHd}]: the discrete gradient vector of storage scalar function $\dxHd{}(\vx, \delta\vx)\in\RR^{\nx}$,
\item[\texttt{hessH}]: the continuous hessian matrix of storage scalar function (computed as $\nabla\nabla\H(\vx)\in\RR^{\nx\times\nx}$),
\item[\texttt{jacz}]: the continuous jacobian matrix of dissipative vector function $\nabla\vz(\vw)\in\RR^{\nw\times\nw}$.
\item[\texttt{y}]: the expression of the continuous output vector function $\vy(\dxH{}, \vz, \vu)\in\RR^{\ny}$,
\item[\texttt{yd}]: the expression of the discrete output vector function $\overline{\vy}(\dxHd{}, \vz, \vu)\in\RR^{\ny}$,
\end{description}
%
\item[\texttt{setexpr(name, expr)}]: Add the \textsc{SymPy} expression \texttt{expr} to the \texttt{exprs} module, with argument 
\texttt{name}, and add \texttt{name} to the set of \texttt{exprs.\_names}.
%
\item[\texttt{freesymbols()}]: Retrun a python set of all the free symbols (\texttt{sympy.symbols}) that appear at least once in all expressions with names in \texttt{exprs.\_names}.
\end{description}
%
\end{description}
%
\item \texttt{dims}
%
\item \texttt{inds}
%
\item \texttt{struc}
%
\item \texttt{exprs}
%
\item \texttt{funcs}
%
\item \texttt{simu}
%
\item \texttt{data}
%
\item \texttt{graph}
%
\begin{itemize}
%
\item 
%
\begin{itemize}
%
\item
%
\end{itemize}
%
\end{itemize}
%
\end{description}
%
%
\section{Port-Hamiltonian system}
%
%
%
%
\begin{equation}
\underbrace{\left(
\begin{array}{c}
\D{\vx}{t}\\
\vw\\
\vy
\end{array}
\right)}_{\vec b}
=
\underbrace{\Big(\J{}-\R{}\Big)}_{\M{}}
\,
\underbrace{\left(
\begin{array}{c}
\dxH{}(\vx)\\
\vz(\vw)\\
\vu
\end{array}
\right)
}_{\vec a}
\end{equation}
%
%
with $\op P=\vu^\T\,\vy$ the power received \emph{by} the sources \emph{from} the system.
%
%
%
\section{Split in linear and nonlinear parts}
%
%
The state is split according to $\vx=(\vxl^\T, \,\vxnl^\T)^\T $ with 
%
\begin{description}
%
\item[$\vxl = (x_1,\cdots,\,x_\nxl)^\T$] the states associated with the quadratic components of the Hamilotnian $\Hl(\vxl)=\vxl^\T\,\Q\,\vxl/2$
%
\item[$\vxnl=(x_{\nxl+1},\cdots,\,x_\nx)^\T$] the states associated with the non-quadratic components of the Hamiltonian with $\nx = \nxl+\nxnl$.
%
\end{description}
%
%
The set of dissipative variables is split according to $\vw = (\vxl^\T, \, \vwnl^\T)^\T$ with 
%
\begin{description}
%
\item[$\vwl = (w_1,\cdots,\,w_\nwl)^\T$] the variables associated with the linear components of the dissipative relation $\vzl(\vwl)=\mat{Z}_{\lab l}\,\vwl$
%
\item[$\vwnl=(w_{\nwl+1},\cdots,\,w_\nw)^\T$] the variables associated with the nonlinear components of the dissipative relation $\vznl(\vwnl)$ with $\nw = \nwl+\nwnl$.
%
\end{description}
%
\begin{equation}
\underbrace{\left(\begin{array}{c}
\D{\vxl}{t}\\
\D{\vxnl}{t}\\
\vwl\\
\vwnl\\
\vy
\end{array}\right)}_{\vec b}
 = \underbrace{\left(\begin{array}{ll|ll|l}
\M{xlxl}&\M{xlxnl}&\M{xlwl} & \M{xlwnl} & \M{xly}\\ 
\M{xnlxl}&\M{xnlxnl}&\M{xnlwl} &  \M{xnlwnl} & \M{xnly}\\ \hline
\M{wlxl}& \M{wlxnl}& \M{wlwl} & \M{wlwnl} & \M{wly}\\
\M{wnlwl}& \M{wnlxnl}& \M{wnlwl} & \M{wnlwnl} & \M{wnly}\\ \hline
\M{yxl} &\M{yxnl}&\M{ywl} & \M{ywnl} &\M{yy}
\end{array}\right)}_{\M{}}
\,
\underbrace{\left(\begin{array}{c}
\nabla\Hl\\
\nabla\Hnl\\
\mat{Z}_{\lab l}\,\vwl\\
\vznl\\
\vu
\end{array}\right) }_{\vec a}
\end{equation}
%
\subsection{Linear subsystem}
%
\begin{equation}
\underbrace{\left(\begin{array}{c}
\D{\vxl}{t}\\
\vwl
\end{array}\right)}_{\vec b_{\lab l}}
 = \underbrace{\left(\begin{array}{ll|ll|l}
\M{xlxl}&\M{xlxnl}&\M{xlwl} & \M{xlwnl} & \M{xly}\\ 
\M{wlxl}& \M{wlxnl}& \M{wlwl} & \M{wlwnl} & \M{wly}
\end{array}\right)}_{\M{l}}
\,
\underbrace{\left(\begin{array}{c}
\nabla\Hl\\
\nabla\Hnl\\
\mat{Z}_{\lab l}\,\vwl\\
\vznl\\
\vu
\end{array}\right) }_{\vec a_{\lab{l}}}
\end{equation}
%
\subsection{Nonlinear subsystem}
%
\begin{equation}
\underbrace{\left(\begin{array}{c}
\D{\vxnl}{t}\\
\vwnl
\end{array}\right)}_{\vec b_{\lab{nl}}}
 = \underbrace{\left(\begin{array}{ll|ll|l}
\M{xnlxl}&\M{xnlxnl}&\M{xnlwl} &  \M{xnlwnl} & \M{xnly}\\ \hline
\M{wnlwl}& \M{wnlxnl}& \M{wnlwl} & \M{wnlwnl} & \M{wnly}
\end{array}\right)}_{\M{nl}}
\,
\underbrace{\left(\begin{array}{c}
\nabla\Hl\\
\nabla\Hnl\\
\mat{Z}_{\lab l}\,\vwl\\
\vznl\\
\vu
\end{array}\right) }_{\vec a_{\lab{nl}}}
\end{equation}
%
\section{Presolve linear part}
%
\subsection{Numerical linear subsystem}
In the sequel, quantities are defined on the current time step $\vx\equiv\vx(t_k)$, with $k\in\mathbb{N}_{+}^{*}$.
%
The dicrete gradient for the quadratic part of the Hamiltonian is ${\small \nabla\Hl =\frac{1}{2}\,\mat{Q}\, (2\vxl + \delta \vxl)}$ and the discret linear subsystem is
 %
%
\begin{equation}
\begin{array}{rcl}
\mat{D}_{\lab l}^{-1} = \mat{iD}_{\lab l} &=& \left(\begin{array}{ll}
\frac{\Id}{\delta t}&0 \\ 
0 &\Id
\end{array}\right)-\left(\begin{array}{ll}
\M{xlxl}&\M{xlwl} \\ 
\M{wlxl}&\M{wlwl} 
\end{array}\right)\,\left(\begin{array}{ll}
\frac{1}{2}\,\mat{Q}&0 \\ 
0 &\mat{Z}_{\lab l}
\end{array}\right),\\
\underbrace{\left(\begin{array}{c}
\delta \vxl\\
\vwl
\end{array}\right) 
}_{\vec{v}_{\lab l}}
& = &
\underbrace{\mat{D}_{\lab l}\,
\underbrace{\left(\begin{array}{ll}
\M{xlxl}\\ 
\M{wlxl}
\end{array}\right)\,\Q}_{\overline{\N{lxl}}}}_{\N{lxl}}
\, 
\vxl
+
  \underbrace{\mat{D}_{\lab l}\,
\underbrace{\left(\begin{array}{ll}
\M{xlxnl}& \M{xlwnl} \\ 
 \M{wlxnl}& \M{wlwnl} 
\end{array}\right)}_{\overline{\N{lnl}}}}_{\N{lnl}}
\,
\underbrace{\left(\begin{array}{c}
\nabla\Hnl\\
\vznl
\end{array}\right) 
}_{\vec{f}_{\lab{nl}}}
+
  \underbrace{\mat{D}_{\lab l}\,
\underbrace{\left(\begin{array}{l}
 \M{xly}\\ 
 \M{wly}
\end{array}\right)}_{\overline{\N{ly}}}}_{\N{ly}}
\,
\vu
\end{array}
\end{equation}
%
\section{Implicite nonlinear function}
%
\subsection{Numerical nonlinear subsystem}
%
\begin{equation}
\begin{array}{rcl}
\left(
\begin{array}{cc}
\frac{\Id}{\delta t} & 0 \\
0 & \Id
\end{array}
\right)\underbrace{\left(\begin{array}{c}
\delta \vxnl \\
\vwnl
\end{array}\right) 
}_{\vec v_{\lab{nl}}}
&=&
  \underbrace{\left(\begin{array}{ll}
\M{xnlxnl}& \M{xnlwnl} \\ 
 \M{wnlxnl}& \M{wnlwnl} 
\end{array}\right)}_{\overline{\N{nlnl}}}
\,
\vec{f}_{\lab {nl}}
+
  \underbrace{\left(\begin{array}{l}
 \M{xnly}\\ 
 \M{wnly}
\end{array}\right)}_{\overline{\N{nly}}}
\,
\vu \\
 && +
\underbrace{ \left(\begin{array}{ll}
\M{xnlxl}&\M{xnlwl} \\ 
\M{wnlxl}&\M{wnlwl} 
\end{array}\right)\,\left(\begin{array}{ll}
\frac{1}{2}\,\mat{Q}&0 \\ 
0 &\mat{Z}_{\lab l}
\end{array}\right)}_{\overline{\N{nll}}}
\, 
\vec v_{\lab{l}}
+
\underbrace{\left(\begin{array}{ll}
\M{xnlxl}\\ 
\M{wnlxl}
\end{array}\right)\,\Q}_{\overline{\N{nlxl}}}
\, 
\vxl
\end{array}
\end{equation}
\subsection{Presolve numerical nonlinear subsystem}
\begin{equation}
\begin{array}{rcl}
\left(
\begin{array}{cc}
\frac{\Id}{\delta t} & 0 \\
0 & \Id
\end{array}
\right)\,
\vec v_{\lab{nl}} &=& \underbrace{\left(\overline{\N{nlxl}} + \overline{\N{nll}}\,{\N{lxl}}\right)}_{\N{nlxl}}\,\vxl + \underbrace{\left(\overline{\N{nlnl}} + \overline{\N{nll}}\,{\N{lnl}}\right)}_{\N{nlnl}}\,\vec f_{\lab{nl}}\\
&& \underbrace{\left(\overline{\N{nly}}+ \overline{\N{nll}}\,{\N{ly}}\right)}_{\N{nly}}\,\vu
\end{array}
\end{equation}
%
\section{Algorithm}
\subsection{Inputs}
\begin{equation}
\begin{array}{rcl}
\mat{iD}_{\lab l} &=& \left(\begin{array}{ll}
\frac{\Id}{\delta t}&0 \\ 
0 &\Id
\end{array}\right)-\left(\begin{array}{ll}
\M{xlxl}&\M{xlwl} \\ 
\M{wlxl}&\M{wlwl} 
\end{array}\right)\,\left(\begin{array}{ll}
\frac{1}{2}\,\mat{Q}&0 \\ 
0 &\mat{Z}_{\lab l}
\end{array}\right) 
\\
\overline{\N{lxl}}
&=&
\left(\begin{array}{ll}
\M{xlxl}\\ 
\M{wlxl}
\end{array}\right)\,\mat Q
\\
\overline{\N{lnl}}
&=&
\left(\begin{array}{ll}
\M{xlxnl} & \M{xlwnl}\\ 
\M{wlxnl} & \M{wlwnl}
\end{array}\right)
\\
\overline{\N{ly}}
&=&
\left(\begin{array}{ll}
\M{xly}\\ 
\M{wly}
\end{array}\right)
\\
\overline{\N{nlnl}}
&=&
\left(\begin{array}{ll}
\M{xnlxnl}& \M{xnlwnl} \\ 
 \M{wnlxnl}& \M{wnlwnl} 
\end{array}\right)
\\
\overline{\N{nll}}
&=&
\left(\begin{array}{ll}
\M{xnlxl}& \M{xnlwl} \\ 
 \M{wnlxl}& \M{wnlwl} 
\end{array}\right)
\,
\left(\begin{array}{ll}
\frac{1}{2}\Q	& 0\\ 
0& \mat Z_\lab l
\end{array}\right)
\\
\overline{\N{nlxl}}
&=&
\left(\begin{array}{ll}
\M{xnlxl}\\ 
 \M{wnlxl}
\end{array}\right)
\,
\Q
\\
\overline{\N{nly}}
&=&
\left(\begin{array}{ll}
\M{xnly}\\ 
 \M{wnly}
\end{array}\right)
\\
\mathcal{J}_{\vec f \lab {nl}}(\vec v_{\lab {nl}})
&=&
\left(\begin{array}{ll}
\mathcal{J}_{\nabla \Hnl}	& 0\\ 
0& \mathcal{J}_{\vznl}
\end{array}\right)
\\
\mat I_{\lab{nl}}
&=&
\left(
\begin{array}{cc}
\frac{\Id}{\delta t} & 0 \\
0 & \Id
\end{array}
\right)
\end{array}
\end{equation}
%
\subsection{Process}
\begin{equation}
\begin{array}{rcl}
\mat D_{\lab l}&=& \mat{iD}_{\lab l}^{-1}\\
\N{lxl} &=& \mat D_{\lab l}\,\overline{\N{lxl}}\\
\N{lnl} &=& \mat D_{\lab l}\,\overline{\N{lnl}}\\
\N{ly} &=& \mat D_{\lab l}\,\overline{\N{ly}}\\
\N{nlxl} &=& \overline{\N{nlxl}} + \overline{\N{nll}}\,{\N{lxl}}\\
\N{nlnl} &=& \overline{\N{nlnl}} + \overline{\N{nll}}\,{\N{lnl}}\\
\N{nly} &=& \overline{\N{nly}} + \overline{\N{nll}}\,{\N{ly}}\\
\vec c &=& \N{nlxl}\,\vxl + \N{nly}\,\vu \\
\mbox{Iterate}  &:&\vec F_{\lab{nl}}(\vec v_{\lab{nl}})=\mat I_{\lab{nl}}\, \vec v_{\lab{nl}}-\N{nlnl}\,\vec f_{\lab{nl}} - \vec c \\
&&\mathcal{J}_{\vec F_{\lab{nl}}}(\vec v_{\lab{nl}}) 
=
\mat I_{\lab{nl}} - \N{nlnl}\,\mathcal{J}_{\vec f \lab {nl}}(\vec v_{\lab {nl}})\\
& & \vec v_{\lab{nl}} = \vec v_{\lab{nl}}-\mathcal{J}^{-1}_{\vec F_{\lab{nl}}}(\vec v_{\lab{nl}}) \,\vec F_{\lab{nl}}(\vec v_{\lab{nl}})\\
\vec v_{\lab{l}} &=& \N{lxl}\,\vxl +\N{lnl}\,\vec f_{\lab{nl}} + \N{ly}\,\vu\\
\vy & = &\M{yxl}\,\nabla \Hl + \M{yxnl}\,\nabla \Hnl\M{ywl}\,\mat Z_{\lab l}\,\vwl + \M{ywnl}\,\vznl + \M{yy}\,\vu\\
\vx & = &\vx+ \delta \vx
\end{array}
\end{equation}

\begin{eqnarray}
\vy & = &\M{yxl}\,\nabla \Hl + \M{yxnl}\,\nabla \Hnl\M{ywl}\,\mat Z_{\lab l}\,\vwl + \M{ywnl}\,\vznl + \M{yy}\,\vu\\
&=& \M{yxl}\,\nabla \Hl + \M{yxnl}\,\nabla \Hnl\M{ywl}\,\mat Z_{\lab l}\,\vwl + \M{ywnl}\,\vznl + \M{yy}\,\vu\\
\end{eqnarray}
\end{document}